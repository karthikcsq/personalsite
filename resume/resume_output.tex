\documentclass[a4paper,11pt]{article}

% Package imports
\usepackage{latexsym}
\usepackage{xcolor}
\usepackage{float}
\usepackage{ragged2e}
\usepackage[empty]{fullpage}
\usepackage{wrapfig}
\usepackage{lipsum}
\usepackage{tabularx}
\usepackage{titlesec}
\usepackage{geometry}
\usepackage{marvosym}
\usepackage{verbatim}
\usepackage{enumitem}
\usepackage{fancyhdr}
\usepackage{multicol}
\usepackage{graphicx}
\usepackage{cfr-lm}
\usepackage[T1]{fontenc}
\usepackage{fontawesome5}

% Color definitions
\definecolor{darkblue}{RGB}{0,0,139}

% Page layout
\setlength{\multicolsep}{0pt} 
\pagestyle{fancy}
\fancyhf{} % clear all header and footer fields
\fancyfoot{}
\renewcommand{\headrulewidth}{0pt}
\renewcommand{\footrulewidth}{0pt}
\geometry{left=1.4cm, top=0.8cm, right=1.2cm, bottom=1cm}
\setlength{\footskip}{5pt} % Addressing fancyhdr warning

% Hyperlink setup (moved after fancyhdr to address warning)
\usepackage[hidelinks]{hyperref}
\hypersetup{
    colorlinks=true,
    linkcolor=darkblue,
    filecolor=darkblue,
    urlcolor=darkblue,
}

% Custom box settings
\usepackage[most]{tcolorbox}
\tcbset{
    frame code={},
    center title,
    left=0pt,
    right=0pt,
    top=0pt,
    bottom=0pt,
    colback=gray!20,
    colframe=white,
    width=\dimexpr\textwidth\relax,
    enlarge left by=-2mm,
    boxsep=4pt,
    arc=0pt,outer arc=0pt,
}

% URL style
\urlstyle{same}

% Text alignment
\raggedright
\setlength{\tabcolsep}{0in}

% Section formatting
\titleformat{\section}{
  \vspace{-4pt}\scshape\raggedright\large
}{}{0em}{}[\color{black}\titlerule \vspace{-7pt}]

% Custom commands
\newcommand{\resumeItem}[2]{
  \item{
    \textbf{#1}{\hspace{0.5mm}#2 \vspace{-0.5mm}}
  }
}

\newcommand{\resumePOR}[3]{
\vspace{0.5mm}\item
    \begin{tabular*}{0.97\textwidth}[t]{l@{\extracolsep{\fill}}r}
        \textbf{#1}\hspace{0.3mm}#2 & \textit{\small{#3}} 
    \end{tabular*}
    \vspace{-2mm}
}

\newcommand{\resumeSubheading}[4]{
\vspace{0.5mm}\item
    \begin{tabular*}{0.98\textwidth}[t]{l@{\extracolsep{\fill}}r}
        \textbf{#1} & \textit{\footnotesize{#4}} \\
        \textit{\footnotesize{#3}} &  \footnotesize{#2}\\
    \end{tabular*}
    \vspace{-2.4mm}
}

\newcommand{\resumeProject}[4]{
\vspace{0.5mm}\item
    \begin{tabular*}{0.98\textwidth}[t]{l@{\extracolsep{\fill}}r}
        \textbf{#1} & \textit{\footnotesize{#3}} \\
        \footnotesize{\textit{#2}} & \footnotesize{#4}
    \end{tabular*}
    \vspace{-2.4mm}
}

\newcommand{\resumeSubItem}[2]{\resumeItem{#1}{#2}\vspace{-4pt}}

\renewcommand{\labelitemi}{$\vcenter{\hbox{\tiny$\bullet$}}$}
\renewcommand{\labelitemii}{$\vcenter{\hbox{\tiny$\circ$}}$}

\newcommand{\resumeSubHeadingListStart}{\begin{itemize}[leftmargin=*,labelsep=1mm]}
\newcommand{\resumeHeadingSkillStart}{\begin{itemize}[leftmargin=*,itemsep=1.7mm, rightmargin=2ex]}
\newcommand{\resumeItemListStart}{\begin{itemize}[leftmargin=*,labelsep=1mm,itemsep=0.5mm]}

\newcommand{\resumeSubHeadingListEnd}{\end{itemize}\vspace{2mm}}
\newcommand{\resumeHeadingSkillEnd}{\end{itemize}\vspace{-2mm}}
\newcommand{\resumeItemListEnd}{\end{itemize}\vspace{-2mm}}
\newcommand{\cvsection}[1]{%
\vspace{2mm}
\begin{tcolorbox}
    \textbf{\large #1}
\end{tcolorbox}
    \vspace{-4mm}
}

\newcolumntype{L}{>{\raggedright\arraybackslash}X}%
\newcolumntype{R}{>{\raggedleft\arraybackslash}X}%
\newcolumntype{C}{>{\centering\arraybackslash}X}%

% Commands for icon sizing and positioning
\newcommand{\socialicon}[1]{\raisebox{-0.05em}{\resizebox{!}{1em}{#1}}}
\newcommand{\ieeeicon}[1]{\raisebox{-0.3em}{\resizebox{!}{1.3em}{#1}}}

% Font options
\newcommand{\headerfonti}{\fontfamily{phv}\selectfont} % Helvetica-like (similar to Arial/Calibri)
\newcommand{\headerfontii}{\fontfamily{ptm}\selectfont} % Times-like (similar to Times New Roman)
\newcommand{\headerfontiii}{\fontfamily{ppl}\selectfont} % Palatino (elegant serif)
\newcommand{\headerfontiv}{\fontfamily{pbk}\selectfont} % Bookman (readable serif)
\newcommand{\headerfontv}{\fontfamily{pag}\selectfont} % Avant Garde-like (similar to Trebuchet MS)
\newcommand{\headerfontvi}{\fontfamily{cmss}\selectfont} % Computer Modern Sans Serif
\newcommand{\headerfontvii}{\fontfamily{qhv}\selectfont} % Quasi-Helvetica (another Arial/Calibri alternative)
\newcommand{\headerfontviii}{\fontfamily{qpl}\selectfont} % Quasi-Palatino (another elegant serif option)
\newcommand{\headerfontix}{\fontfamily{qtm}\selectfont} % Quasi-Times (another Times New Roman alternative)
\newcommand{\headerfontx}{\fontfamily{bch}\selectfont} % Charter (clean serif font)

\begin{document}
\headerfontiii

% Header
\begin{center}
    {\Huge\textbf{KARTHIK THYAGARAJAN}}
\end{center}

\begin{center}
    \small{
    \href{https://www.karthikthyagarajan.com}
    {karthikthyagarajan.com} | \href{karthik6002@gmail.com}{karthik6002@gmail.com} | 
    \href{kthyagar@purdue.edu}{kthyagar@purdue.edu} | 
    \socialicon{\faLinkedin} \href{https://www.linkedin.com/in/karthikthyagarajan06}{linkedin.com/in/karthikthyagarajan06/} | 
    \socialicon{\faGithub} \href{https://github.com/karthikcsq}{github.com/karthikcsq} 
    }
\end{center}

\section{\textbf{Education}}
\resumeSubHeadingListStart


  \resumeSubheading
    { Purdue University }
    { West Lafayette, Indiana }
    { B.S. of Computer Science \& Artificial Intelligence - 4.0 GPA }
    { Currently Pursuing }


\resumeSubHeadingListEnd

\vspace{-4mm}

\section{\textbf{Skills}}
\resumeHeadingSkillStart

  \resumeSubItem{ AI/ML : }{ Deep Learning, LLM (LangChain, RAG Implementation, CoT/Reasoning Models, RLHF, Decision-Making), Agents, MCP, PyTorch, Tensorflow, Adversarial Learning (GANs), RL, Diffusion, Graph Neural Networks }

  \resumeSubItem{ Data Science : }{ Numpy, Pandas, PostgreSQL, NoSQL }

  \resumeSubItem{ Languages \& Frameworks : }{ Python, Java, C++, C, JavaScript, TypeScript, Next.js, React, Flask, Gradle }

  \resumeSubItem{ Quantum Computing : }{ Qiskit, VQE, QAOA, Quantum ML }

  \resumeSubItem{ Other : }{ REST APIs, SDK Development, AWS, Google Cloud, OAuth, Git, Docker, Linux }

\resumeHeadingSkillEnd

\section{\textbf{Experience}}
\resumeSubHeadingListStart

  \resumeSubheading
  { Machine Learning Researcher }
  { Silver Spring, MD }
  { Peraton Labs \( Internship \& Part-Time Co-op \) }
  { June 2025 - Present }
  \resumeItemListStart
  
    \item Designed and implemented the first-of-its-kind reinforcement-learning agent to intelligently navigate complex IoT device environments—reducing exploration latency by 35\\% and boosting malware detection coverage by 25\\% over a brute-force search baseline.
  
    \item Implemented a heterogeneous graph neural network combined with autoencoders to model inter-device relationships and inform RL agent policy, aiming to enhance policy convergence and improve malware detection accuracy.
  
  \resumeItemListEnd

  \resumeSubheading
  { Computer Vision Researcher }
  { Remote }
  { Memories.ai \( Part-Time \) }
  { February 2025 - August 2025 }
  \resumeItemListStart
  
    \item Engineered and deployed a customer-facing, scalable video memory framework for AR apps, enabling long-term spatial and contextual awareness and optimizing throughput for speed and scale.
  
    \item Designed and launched a Python SDK for the Mavi platform, driving end-to-end developer workflows around video analysis; published to PyPI at https://pypi.org/project/pymavi/.
  
  \resumeItemListEnd

  \resumeSubheading
  { Undergraduate Robotics Researcher }
  { West Lafayette, IN }
  { IDEAS Lab at Purdue University \( Part Time \) }
  { March 2025 - June 2025 }
  \resumeItemListStart
  
    \item Collaborated in a cross-functional, iterative team of engineers and researchers to build end-to-end real-time SLAM and novel view-synthesis pipelines, improving scene reconstruction accuracy by 25\\% while ensuring safety and reliability in deployment.
  
    \item Implemented performance optimizations in C++ and Python to scale mapping algorithms for autonomous navigation, reducing processing latency.
  
  \resumeItemListEnd

  \resumeSubheading
  { Undergraduate Data Science Researcher }
  { West Lafayette, IN }
  { The Data Mine Corporate Partners \( Part Time \) }
  { August 2024 - December 2024 }
  \resumeItemListStart
  
    \item Partnered with AgRPA in an iterative, cross-disciplinary team to build an end-to-end weed-detection pipeline using Python, TensorFlow, and Postgres, optimizing database queries for 40\\% faster retrieval.
  
    \item Developed semantic segmentation and localization models to accurately locate weeds during real-time drone flight, speeding up ground-vehicle-based methods by 50\\% and cutting herbicide costs by 60\\%.
  
  \resumeItemListEnd

  \resumeSubheading
  { ML Science and Engineering Apprenticeship (SEAP) Intern }
  { Washington, D.C. }
  { Naval Research Laboratory \( Full Time \) }
  { June 2023 - August 2023 }
  \resumeItemListStart
  
    \item Led a team of four in applying machine-learning models (UNets, Transformers, GANs) to underwater acoustics, improving transmission loss prediction accuracy by 20\\% compared to physics modeling.
  
    \item Developed and deployed an internal, secure Retrieval-Augmented Generation prototype, maintaining data confidentiality and meeting reliability SLAs.
  
  \resumeItemListEnd

\resumeSubHeadingListEnd

\section{\textbf{Projects}}
\resumeSubHeadingListStart

\resumeProject
  { Personal Website }
  {Tools: Next.js, React, TypeScript, Tailwind CSS, Vercel, Pinecone, AWS S3, Python }
  { Ongoing }
  { \href{ https://github.com/karthikcsq/personalsite }{ https://github.com/karthikcsq/personalsite } }
\resumeItemListStart
  
  \item Architected and maintained https://www.karthikthyagarajan.com with Next.js, React, TypeScript \& Tailwind CSS; deployed via Vercel and AWS S3 (secure buckets + pre-signed URLs) to host portfolio, blog \& image gallery, leveraging Vercel’s CDN for performant global delivery.
  
  \item Built a Pinecone-backed RAG pipeline with Python scripts to power searchable Markdown docs of my projects and work history.
  
\resumeItemListEnd

\resumeProject
  { Verbatim }
  {Tools: OpenAI/Google Cloud APIs, Miscellaneous APIs, Next.js, Vercel }
  { February 2025 }
  { \href{ https://github.com/TheXDShrimp/verbatim }{ https://github.com/TheXDShrimp/verbatim } }
\resumeItemListStart
  
  \item Developed a platform to summarize, translate, voice clone, and lip sync any video, deployed using Vercel at https://www.getverbatim.tech.
  
  \item Created an automated pipeline with audio transcription (Whisper), translation (Google Cloud), summarization (GPT-4o), automatic voice cloning (Eleven Labs), lip sync (Sync.so) and video Q\&A (Twelve Labs) for enhanced interactivity.
  
\resumeItemListEnd

\resumeProject
  { Photonic Implementation of Quantum Key Distribution }
  {Tools: Oscilloscope, Scientific Computing, Python Data Parsing, Numpy }
  { October 2023 - May 2024 }
  { \href{ https://www.karthikthyagarajan.com/QKDResearchPoster.pdf }{ https://www.karthikthyagarajan.com/QKDResearchPoster.pdf } }
\resumeItemListStart
  
  \item Engineered and assembled a lab-scale photonic QKD prototype, leveraging a 650 nm laser, inline polarizers, servo-controlled phase modulators, single-mode fibers, and polarizing beamsplitters, to implement H/V and D/AD polarization encoding; optimized optical alignment and polarization fidelity to generate a sifted key that matched theoretical predictions.
  
  \item Developed Python pipelines for oscilloscope waveform thresholding (0.004 mW cutoff), bit-sequence extraction, and basis sifting; conducted noise and signal-loss analysis.
  
\resumeItemListEnd

\resumeProject
  { Quantum Racer (Android Educational Game) }
  {Tools: Java, Android SDK, Gradle, XML Layouts, Game Physics }
  { August 2022 - December 2022 }
  { \href{ https://github.com/karthikcsq/QuantumCarGame\_Self }{ https://github.com/karthikcsq/QuantumCarGame\_Self } }
\resumeItemListStart
  
  \item Architected and implemented a 100\% Java-based Android game using Android SDK, Gradle and Git, translating core quantum mechanics concepts—superposition, measurement, probability distributions, noise and decoherence—into interactive racing mechanics; built and packaged the APK for side-loading and future Play Store release.
  
  \item Owned end-to-end development: game physics and state management, touch-input and UI flow via XML layouts, asset pipeline (PNG sprites), performance tuning on mobile devices, and comprehensive user documentation to guide educational outreach.
  
\resumeItemListEnd

\resumeSubHeadingListEnd
\end{document}